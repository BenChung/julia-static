\documentclass{report} % can add review if you want line numbers
\usepackage{stmaryrd}
\usepackage{xspace,listings,url,framed,amssymb,
            mathpartir,hyperref,doi, mathtools,wrapfig,
            stmaryrd, graphicx, tikz, colortbl, xparse, etoolbox,
            pgffor, makecell, amsmath, amssymb} % double brackets llbracket

\begin{document}
	
\title{TBD}
\maketitle	
	
% todos
% Ben
%   Come up with a declarative specification for the type system 
% Paley
%   Subtyping and WF for declarative system.

\section*{Motivating example}

\begin{verbatim}
function foo(x::Int)::String
	return string(x)
end

function foo(x::String)::Int
	return parse(Int, x)
end

function baz{T}(x::T)::T
    return x
end

function caz{T}(x::T)::T
	return "hello"
end

function haz{T}(x::T, y::T)::T
	return x
end

function haz{T}(x::T, y::Array{T})::T
	return x
end

function bar()
    println(foo(2) * "baz") # 2baz
    println(foo("2") + 2) # 4
    println(baz("2") * "baz") # hellobaz
    println(baz(2) + 2) # error
    foo("2") * "baz" # error
    haz(3, "hello") # error
    caz("hi") # hello
    caz(2) # error
end

\end{verbatim}


\section*{Declarative type system}

\begin{verbatim}
# function table
 ft ::= function func_name(arg_name) : tf = fb | \empty

# type hierarchy - no cycles, no multiple inheritance. 
# have to be a tree, and the leafs are concrete types.
 th := name <: t th | empty

# types
 t ::= any | name{t, ...} | T | (t, ...)

# function types are
 tf ::= tf where T | t -> t

# environment
 psi ::= x : t psi | T psi | empty

# function bodies
 fb ::= e ; fb | v = e ; fb | e 
# are function bodies allowed to be empty? no, but we'll deal in impl
# how do you write a function bodies with just v = 1? not possible, impl

#expressions
 e ::= v (DeclVar)
     | fun(e) (Decl->E to DeclForallApp and Decl->App)
     | name{t,...}() (Decl->E)
     | e :: t (DeclAnno)
     | (e, ...) (below)

#subtyping

#psi/E duality

Francesco's paper
TUPLE, APP_INV, APP_SUPER, L_LEFT, L_RIGHT, R_LEFT, R_RIGHT, 
REFL, TOP

# assignment typing
\inferrule*[]{
     psi |- e => tv
     psi,v:tv |- fb <= t
}{
     psi |- v = e ; fb <= t
}

# sequence typing
\inferrule*[]{
     psi |- e1 => t1
     psi |- fb <= t
}{
     psi |- e1 ; fb <= t
}

# tuple typing
     psi |- e1 => t1  ...  psi |- en => tn
    -------------------------------------
     psi |- (e1,...,en) => (t1,...,tn)

     psi |- e1 <= t1  ...  psi |- en <= tn
    --------------------------------------
     psi |- (e1,...,en) <= (t1,...,tn)

# cov metafunction

# cov(any, T) = 0
# cov(Tuple{t1,...,tn},T) = sum_i=1^n cov(ti, T)
# cov(name{t1,...,tn},T) = 0
# cov(T, T) = 1
# cov(U, T) = 0 if U != T

# forall types

     cov(t,T) <= 1   th,psi |- t'  th,psi |- [t'/T]t • e =>> t''
    -----------------------------------------------------------
              th, psi |- t where T • e =>> t''

       cov(t,T) > 1   th,psi |- t'  th |- t' concrete  
               th,psi |- [t'/T]t • e =>> t''
    -----------------------------------------------------
              th,psi |- t where T • e =>> t''

# no bound mods needed, triangle rule makes unsound in all cases
# Bound (intro in covariant position) gets overwritten by invariant

# Subtyping - types & function types



# Well-formedness - types, function table (function types & function body),
#                 - type hierarchy

     T \in psi
   ---------------
     psi |- T \ok

     
   -----------------
     psi |- any \ok

     name in th   psi |- t1 \ok ...
   -----------------------------------
     th, psi |- name{t1, ...} \ok

     psi |- t1 \ok ...
   ------------------------
     psi |- (t1, ...) \ok 

     \forall t . name1 <: t \in th | th, psi |- t \ok 
     ## checks for th - non cyclic, no multi hier.
   --------------------------------------------------
     psi |- th \ok 


# algo

# envs
# gamma etc := x : t | T_hat | T_hat <: t (cov) | T_hat = t (inv) | T | mark_T 
# omega := x : t | T_hat = t | T | mark_T

\end{verbatim}
\end{document}